\documentclass[10pt,a4paper]{article}
\usepackage[utf8]{inputenc}
\usepackage[T1]{fontenc}
\usepackage[margin=2cm]{geometry}
\usepackage{mdwlist}
\usepackage{url}

\begin{document}
\title{On the philosophy of 'Clean Data'}
\author{October 2016}
\date{}
\maketitle
\setlength{\parindent}{0cm}

\begin{abstract}
  As GA's instructors eloquently put it: ``Good models cannot produce
  good predictions without good data.'' This paper describes how
  important it is to have clean data if you want to do data science.
\end{abstract}

%%%%%%%%%%%%%%%%%%%%%%%%%%%%%%%%%%%%%%%%%%%%%%%%%%%%%%%%%%%%%%%%%%%%%%%%%%%%%%%%
\section{What is data}

According to the Wikipedia (\url{https://en.wikipedia.org/wiki/Data}),
``Data is a set of values of qualitative or quantitative variables. An
example of qualitative data would be an anthropologist's handwritten
notes about her interviews with people of an Indigenous tribe.''
 
Data is important if we want to gather information about the world to
make informed decisions.  For example, if we decide to eat a healthy
diet and want to choose healthy foods, we need data to choose wich
foods are healthier than others.  If we decide to invest our saving
for retirement, we need data to tell us which investments offer bigger
returns given the risks that we are willing to take.  If we want to
drive to work, we need to know which roads to take, and if possible
avoid the ones that have more traffic.  All these tasks need data in
one form or another.

%%%%%%%%%%%%%%%%%%%%%%%%%%%%%%%%%%%%%%%%%%%%%%%%%%%%%%%%%%%%%%%%%%%%%%%%%%%%%%%%
\section{Cleaning data}

There are a few cleaning data tasks that we have done this week.

\begin{enumerate}
\item Fix data formats: sometimes data are stored as strings when they
  should be dates, for example.  In this case, it is necessary to
  correctly interpret these values, for example using
  \texttt{.to\_datetime()}.  When there are missong values in columns
  that are supposed to be numbers, they will appear as strings and
  need to be converted.
\item Fill in missing values: this can be done simply by substituting
  missing strings with empty strings and missing numbers with zeros.
  However, sometimes we need to follow other approches, for example
  substitute missing numbers with the mean of the remaining numbers.
\item Correct erroneous values: sometimes values in a column are so
  far outside the expected values that they can be attributed to an
  error and must be corrected.  For example, an age of 567 years, or a
  gender of 27.  We can simply delete those values or try to find
  approximations for them, either in the same dataset or from other
  sources.
\item Standardise categories: sometimes, it's useful to have a
  pre-defined set of categories to use with our data, for example, a
  pre-defined set of music genres.  However, if the genre is entered
  manually, we may find a few that are outside our categories (for
  example due to spelling mistakes) in this case, this needs to be
  corrected.
\item Eliminate superfluous rows/columns: sometimes, we have a huge
  dataset, but we only need some of the data to draw our conclusions.
  Or we may have rows full of missing values.  In this case, it is
  useful to drop useless data and make our dataset smaller and more
  manageable.
\end{enumerate}

%%%%%%%%%%%%%%%%%%%%%%%%%%%%%%%%%%%%%%%%%%%%%%%%%%%%%%%%%%%%%%%%%%%%%%%%%%%%%%%%
\section{Conclusion}

There is much more to be said about this subject, but it does not fit in
500 words.  These are just some of the most fundamental ideas.

\end{document}

%%%% Local Variables: 
%%%% mode: latex
%%%% ispell-local-dictionary: "en_UK"
%%%% TeX-master: t
%%%% End: 
